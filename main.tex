\documentclass[twocolumn]{article}

\title{12:20}
\author{Ethan Wu}
\date{\today}

\usepackage{ccicons}
\begin{document}
\maketitle

\section{Introduction}
12:20 (``twelve-twenty'') is an extremely lightweight narrative system intended
for use as a side activity alongside some other tabletop role-playing game.

\section{The 12:20 Check}
When a character in a 12:20~scene wishes to take an action that has a chance of
failure, they must make a \emph{12:20~check}.

The character receives two~$d12$s. These are their \emph{attempt dice}. They
also receive a~$d20$. This is the \emph{difficulty die}. There are only ever
two attempt dice and one difficulty die in play for any given check.
Additionally, beneficial circumstances may grant the character \emph{bonus
dice}, and detrimental circumstances may confer \emph{penalty dice}. The
entire dice pool is rolled together, and the total of the attempt and bonus dice
are compared against the total of the difficulty and penalty dice. If the total
of the attempt and bonus dice is greater, the check results in a success.
Otherwise, the check results in a failure.

\subsection{Bonus and Penalty Dice}
\emph{Bonus dice} are granted by circumstances that improve the character's
ability to succeed at a check. For each positive circumstance, add one~$d6$ to
the check's bonus dice.

\emph{Penalty dice} are conferred by circumstances that significantly reduce a
character's ability to succeed at a check. For each negative circumstance, add
one~$d10$ to the check's penalty dice. There may only be six penalty dice for
any given check.

If a character has proficiency in an applicable skill, they may apply one skill
towards their 12:20~check, adding as many bonus dice as they have proficiency
levels, up to~4. If skill proficiencies are not leveled, but are derived from
some other modifier, they may add a number of bonus dice equal to their
modifier. If a 12:20~check is being made against a character that is given the
opportunity to defend, that character may apply skill proficiencies to add
penalty dice to the check. Finally, an ally may apply one of their own skill
proficiencies to assist or hinder a 12:20~check, adding a single bonus or
penalty die, respectively.

\subsection{Caveats}
\emph{Caveats} are narrative effects that influence the result of a check. If
every two attempt dice that roll higher than the difficulty die, add an
\emph{advantage}. For every two attempt dice that roll lower than the difficulty
die, add a \emph{setback}. One advantage and one setback cancel each other out.

If a character faces more than two penalty dice, they may opt to \emph{Take a
Risk} and remove two penalty dice, downgrading their attempt dice to two~$d8$s.
This increases their chances of success at the cost of dramatically increasing
the probability of a setback. If any penalty dice remain, they may opt to
\emph{Take a Greater Risk} and remove two more penalty dice, downgrading their
attempt dice to two~$d4$s.

\section{Acknowledgements}
This work is inspired by \emph{Pathfinder, Second Edition} and \emph{Genesys}.

\section{License}
\textcopyright{} 2022 Ethan Wu. This work is licensed under a Creative Commons
Attribution-ShareAlike~4.0 license. \ccbysa

\end{document}
